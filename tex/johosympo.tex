% written by Kei YASUDA 2017
% 応募要領指定フォントに準ずるために,xelatexを用いる。
\documentclass[base=10pt,magstyle=real,a4paper,twocolumn,xelatex,pandoc,jafont=ms]{bxjsarticle}
\usepackage{aij_johosympo_xelatex}


\usepackage{hyperref}

%タイトル,著者,キーワードなどを記入
\title{駅前広場における景観の多様性とその評価に関する研究}
\subtitle{サブタイトル}
\foreigntitle{DIVERSITY AND FAVORABILITY OF TOWNSCAPES IN STATION PLAZAS}
\foreignsubtitle{Sub title}
\author{構造 一郎\textsuperscript{*1}
	          ,環境 二郎\textsuperscript{*2}
	          ,計画 三郎\textsuperscript{*3}
             }
\foreignauthor{Ichirou KOUZOU
			             Jirou KANKYO
	                     and Saburo KEIKAKU
             }%英文での著者表記
\affiliation{
	\mbox{*1}&建築工業大学工学部建築学科 教授・工博\\
	\mbox{*2}&建築工業大学工学部建築学科 助手・工修\\
	\mbox{*3}&建築工業大学工学部建築学科 大学院生・工修
}%アスタリスクは本文で打っても数式で打っても文字間などうまくいかない。
\ekeyword{Diversity Favorability, Single-view-point townscape, 4-view-points townscape, Station plaza}
\jkeyword{多様性,好ましさ,単一視点景観,4視点景観,駅前広場,レパートリー}

%%%---definition begin---%%%
\newcommand{\marker}[1]{\underline{\textbf #1}}

\usepackage{verbatim, setspace,framed}
\newcommand{\inputpython}[1]{
	\begin{oframed}
		\begin{spacing}{0.72}
			{\small 
				\texttt{\verbatiminput{#1}}
			}
		\end{spacing}
	\end{oframed}
}

%%%---definition end---%%%
\pagestyle{fancy}
%\thispagestyle{fancy}
\lhead{}
\chead{}
\rhead{}
\lfoot{報告 H00}
\cfoot{}
\rfoot{}


%ここから本文
\begin{document}

%\begin{abstract}
%	Diversity and favorability of townscapes in station plazas were analysed on following 3 aspects.
%	\begin{enumerate}
%		\renewcommand{\labelenumi}{\arabic{enumi}. }%ここだけ箇条書きの数字を変更する
%		\item 	Relations between evaluation and physical elements composed by means of repertory grid method were analyzed and physical elements were classified into; favorable elements unfavorable elements fuzzy elements.
%		
%		\item By using SD method with 40 couples of adjectives and factor analysis, 2 independent axes of “favorability” and “diver- sity” were shown. Upon 2 dimensional plane, 4 areas were translated as meanings of; “fascinatingness”, “orderlyness”, “disorderlyness”, “boringness”.
%		
%		\item Indivisual pictures by single-view-point, composed pictures by 4-view-points, and their relations were located upon 2 dimensional plane of; “favorable-unfavorable”, “diverse-monotonous”. 
%	\end{enumerate}s
%	
%\end{abstract}
\maketitle %abstractの後にかいて,abstractを段組にしない
\thispagestyle{fancy}
\pagestyle{fancy}%draftの場合,ヘッダーを表示するとよい。
%\pagestyle{empty}%emptyにするとヘッダーは表示されない。
%\tableofcontents %目次


%\section{序論}
%	\subsection{背景}
%	\url{http://www.aij.or.jp/jpn/transaction/ronbun/}
%

\section{はじめに}
	論文作成にあたり「版下原稿執筆の手引き」を参照して,「論文集執筆要領」の各条項に従って作成願います。
	見本として入力されている文字の間に文字を入力して,入力後に不要文字を削除すると組体裁がそのまま残ります。
	先に削除するとテンプレートがなくなることがあります。
	
	本文の書体「和文:MS明朝、章節の表題はMSゴシック/英文:Century、章節の表題はArial」,文字の大きさ8pt,2段組み,1行30字,段間2字(6㎜),行間14pt 50行,1頁(30文字×50行×2段=3,000字詰)、余白 上20mm 下30mm 左右15mm。

\section{タイトル・著者名・英文要旨・キーワードについて}
	本文が和文の場合は和文タイトルを先に,その下行に英文タイトルを記載してください。
	本文が英文の場合は英文タイトルを先に,その下行に和文タイトルを記載してください。
	著者名も同様です。
	
	\begin{enumerate}
		\renewcommand{\labelenumi}{$ \textcircled{\scriptsize\arabic{enumi}} $}%ここだけ箇条書きの数字を変更する
		\item 先のタイトル 14pt MS明朝 英数はCentury 英文の場合はすべて大文字/サブタイトル 10.5pt MS明朝 英数はCentury 中央揃え。英文の場合は最初の1語のみキャピタルラージとしています。
		
		\item 下行のタイトル 10.5pt MS明朝 英数はCentury 英文の場合はすべて大文字/サブタイトル 9pt MS明朝 英数はCentury 中央揃え。英文の場合は最初の1語のみキャピタルラージとしています。
		
		\item 和文著者名 10.5pt MS明朝 文字均等割付5字,氏名が3字以下は名字と名前の間に全角スペースを入れて均等割付5字,5字以上は名字と名前の間に半角スペースを入れてください。中央揃え,*(合い印)は全角*を上付きにしてください。
		
		\item 英文著者名(\textit{Namae MYOUZI})  10.5pt \textit{Century Italic}  
		
		\item 英文要旨(Abstract) 8pt Century 行間1行(14pt相当) ,左右インデント 各4字。	
		
		\item {\timesnewroman\textit {\textbf{Keywords:~}}} の見出しは9pt {\timesnewroman\textit {\textbf{Times New Roman Italic Bold}}}としてください。

		\item 英文キーワードは 8pt Times New Roman Italic 行間1行(14pt相当) 左右インデント 各4字 中央揃え。

		\item 和文キーワードは 8pt MS明朝 行間1行(14pt相当) 左右インデント 各2字 中央揃え。
		
	\end{enumerate}

\section{質疑討論(回答)の書き方について}
「版下原稿執筆の手引き」(組見本B)を参照してください。
質疑討論では,対象とする論文の著者名・論文名・号数・掲載年月日を先のタイトル,下行のタイトルともにカッコ付きサブタイトルとして記載してください。回答では、英文サブタイトルのみに記載してください。

\section{見出し}
(1) 見出しは 8pt MSゴシック 英文はArial

\section{第1頁の脚注について}
脚注は組体裁の都合上Wordの「脚注」機能を使わずに表を使用して,罫なしで作成しています。
(1) 和文
文字の大きさは 7pt  MS明朝 行間=固定値10pt
(2) 英文
文字の大きさは 7pt Century  行間=固定値10pt
(3) 表の「列」幅
本文が和文の場合は,表の「列」幅は和文(左側 70.5㎜),英文(右側 97.5㎜)としています。本文が英文の場合は,左右入れ替えます。英文(左側 97.5㎜),和文(右側 70.5㎜)としています。

\section{表について}
表と本文の間は1行空け中央揃えにしてください。
表題は,表の上に罫なしのセルを作りその中に入れると,表と表題がバラバラになりません。
表番および表題はMSゴシック(英文はArial)で英語表記を推奨する。

\begin{table}[H]
\caption{The target station square}
\small
\centering
\begin{tabular}{p{35mm}|p{35mm}}
	\hline 
	A name of the station and an exit name
 & The form of an open space
\\ 
	\hline \hline
	Chuo-rinkan & Rotary\\
	Saginuma & Rotary\\
	Tsukimino & Rotary\\
	Miyazaki-dai& Pedestrian space\\
	Minami-Machida& Pedestrian space\\
	Futako-tamagawaen (west)&Pedestrian space\\
	Suzukake-dai&Pedestrian space\\
	Jiyugaoka&Rotary\\
	Tsukushino&Pedestrian space\\
	Yutenji&Rotary\\
	Nagatsuta&Rotary\\
	Den-enchofu&Rotary\\
	Ichigao&Pedestrian space\\
	JR Musashikosugi&Rotary\\
	Eda&Rotary\\
	Tokyu Musashikosugi&	Pedestrian space\\
	Azamino (east)&Pedestrian space\\
	Hiyoshi (south)&Rotary\\
	Azamino (west)&Rotary\\
	Sakuragicho&Pedestrian space\\
	Tama-plaza&Pedestrian space\\
	Ikegami&Rotary\\
	\hline 
\end{tabular} 
\end{table}

\section{図について}
本文と図(\figref{fig:figure1})の間は1行空け中央揃えにしてください。
図番および図題はMSゴシック(英文はArial)で英語表記を推奨する。
\begin{figure}[h]
	\centering
	\includegraphics[width=1\linewidth]{../figure/figure1}
	\caption{Photography point}
	\label{fig:figure1}
\end{figure}


\section{写真について}
 写真(\figref{fig:photo1})と本文の間は1行空け中央揃えにしてください。
写番および写題はMSゴシック(英文はArial)で英語表記を推奨する。

\begin{figure}[h]
	\centering
	\includegraphics[width=1\linewidth]{../figure/photo1}
	\caption{An example of the photograph B}
	\label{fig:photo1}
\end{figure}


\section{カッコ付き}
箇条書き番号はカッコと数字を半角にして,カッコ・数字ともに書体をMS明朝にするときれいに揃います。
 例:(1) 1) 1)(右側の片カッコの1は全角)
\footnote{注表題は 7pt MSゴシック(英文はArial) 上1行アキ}
\footnote{注の文字の大きさは 7pt MS明朝(英文はCentury) 行間 10.5pt 複数行は1字下げてください。番号は片カッコ付きで数字は半角にしてください。}

\section{英文要約または和文要約について}
 英文要約(600語以内)は本文の書体で論文の末尾に改頁し付ける。和文要約(3,000字以内)は本文の書体で論文の末尾に付ける。


\vspace{1\Cvs}
\begin{thebibliography}{9}
	\bibitem{参考文献1}
		参考文献表題は 7pt MSゴシック(英文はArial) 上1行アキ
	\bibitem{参考文献2}
		参考文献の文字の大きさは 7pt MS明朝(英文はCentury) 行間10.5pt 複数行は1字下げてください。番号は片カッコ付きで数字は半角にしてください。


\end{thebibliography}
\vspace{1\Cvs}
%後注を出力
\theendnotes

\noindent 1234567890
1234567890
12345
1\\2\\3\\4\\5\\6\\7\\8\\9\\10\\
1\\2\\3\\4\\5\\6\\7\\8\\9\\20\\
1\\2\\3\\4\\5\\6\\7\\8\\9\\30\\
1\\2\\3\\4\\5\\6\\7\\8\\9\\40\\
1\\2\\3\\4\\5\\6\\7\\48\\
1\\2\\3\\4\\5\\6\\7\\8\\9\\10\\
1\\2\\3\\4\\5\\6\\7\\8\\9\\20\\
1\\2\\3\\4\\5\\6\\7\\8\\9\\30\\
1\\2\\3\\4\\5\\6\\7\\8\\9\\40\\
1\\2\\3\\4\\5\\6\\7\\48\\
1\\2\\3\\4\\5\\6\\7\\8\\9\\10\\
1\\2\\3\\4\\5\\6\\7\\8\\9\\20\\
1\\2\\3\\4\\5\\6\\7\\8\\9\\30\\
1\\2\\3\\4\\5\\6\\7\\8\\9\\40\\
1\\2\\3\\4\\5\\6\\7\\48\\
1\\2\\3\\4\\5\\6\\7\\8\\9\\10\\
1\\2\\3\\4\\5\\6\\7\\8\\9\\20\\
1\\2\\3\\4\\5\\6\\7\\8\\9\\30\\
1\\2\\3\\4\\5\\6\\7\\8\\9\\40\\
1\\2\\3\\4\\5\\6\\7\\48\\

\noindent \rule[0.5em]{\columnwidth}{0.5truept}\\
\makeaffiliation


\onecolumn %英文要約または和文要約の為に1段組に戻す
\begin{center}
	{\large \ftitle}\\
	{\normalsize \fsubtitle}\\

	\vskip 1.5em
	{\large\textit{Ichirou KOUZOU\textsuperscript{*}
				 Jirou KANKYO\textsuperscript{**}
				 and Saburo KEIKAKU\textsuperscript{***}}}
	\vskip 1.5em
\end{center}
\bgroup
\leftskip=10truemm \rightskip=10truemm

There are several types of storage tanks, e.g., above-ground, flat-bottomed, cylindrical tanks for the storage of refrigerated liquefied gases, petroleum, etc., steel or concrete silos for the storage of coke, coal, grains, etc., steel, aluminium, concrete or FRP tanks including elevated tanks for the storage of water, spherical tanks (pressure vessels) for the storage of high pressure liquefied gases, and under-ground tanks for the storage of water and oil. 
The trend in recent years is for larger tanks, and as such the seismic design for these larger storage tanks has become more important in terms of safety and the environmental impact on society as a whole.

The failure mode of the storage tank subjected to a seismic force varies in each structural type, with the structural characteristic coefficient (Ds) being derived from the relationship between the failure mode and the seismic energy transferred to, and accumulated in the structure. A cylindrical steel tank is the most common form of storage tank and its normal failure mode is a buckling of the cylindrical shell, either in the so called Elephant Foot Bulge (EFB), or as Diamond Pattern Buckling (DPB). The Ds value was originally calculated with reference to experimental data obtained from cylindrical shell buckling, but was later re-assessed and modified based on the restoring force characteristics of the structure after buckling. Those phenomena at the Hanshin-Awaji Great Earthquake and the Niigataken Chuetu-oki Earthquake were the live data to let us review the Ds value. For the EFB, which is the typical buckling mode of a cylindrical shell storage tank for petroleum, liquefied hydrocarbon gases, etc., it became possible to ascertain the buckling strength by experiments on a cylindrical shell by applying an internal hydrodynamic pressure, an axial compressive force, and a shear force simultaneously. Details of these experiments are given in Chapter 3.

The seismic design calculations for other types of storage tanks have been similarly reviewed and amended to take into account data obtained from recent experience and experiments.

Design recommendation for sloshing phenomena in tanks has been added in this publication. Design spectra for sloshing, spectra for long period range in other words, damping ratios for the sloshing phenomena and pressures by the sloshing on the tank roof have been presented.

For above-ground vertical cylindrical storage tanks without any restraining element, such as anchor bolts or straps, to prevent any overturning moment, only the bending resistance due to the uplift of the rim of bottom plate exists. This recommendation shows how to evaluate the energy absorption value given by plasticity of the uplifted bottom plate for unanchored tanks, as well as the Ds value of an anchored cylindrical steel-wall tank.

As the number of smaller under-ground tanks used for the storage of water and fuel is increasing in Japan, the Sub-committee has added them in the scope of the recommendation and provided a framework for the seismic design of under-ground tanks. The recommendation has accordingly included a new response displacement method and a new earth pressure calculation method, taking into account the design methods adopted by the civil engineering fraternity.

For silo design, additional local pressure which depends on eccentricity of discharge outlet, and equations which give approximate stress produced by this pressure are given in this 2010 publication.

\par
\egroup

\noindent \rule[0.5em]{\columnwidth}{0.5truept}\\
{\footnotesize
	\begin{tabular}{rl}
		\mbox{*1}&Prof., Dept. of Architecture, Faculty of Engineering, Kenchiku Institute of Technology, Dr. Eng\\
		\mbox{*2}&Research Assoc. Dept. of Architecture Faculty of Engineering Kenchiku Institute of Technology M. Eng.\\
		\mbox{*3}&Graduate Student Dept. of Architecture Faculty of Engineering Kenchiku Institute of Technology M. Eng.
	\end{tabular}
}

\end{document}          
