% written by Kei YASUDA 2017
% 応募要領指定フォントに準ずるために,xelatexを用いる。
\documentclass[base=11pt,magstyle=real,a4paper,twocolumn,xelatex,pandoc,jafont=ms]{bxjsarticle}
\usepackage{aij_kinki_xelatex}

%タイトル,著者,キーワードなどを記入
\title{駅前広場における景観の多様性とその評価に関する研究}
\subtitle{サブタイトル}
\foreigntitle{DIVERSITY AND FAVORABILITY OF TOWNSCAPES IN STATION PLAZAS}
\foreignsubtitle{Sub title}
\author{正会員\hspace{-.1pt}$\bigcirc$\hspace{-.1pt}構造 一郎\textsuperscript{*1}\quad
	          同\quad 環境 二郎\textsuperscript{*2}\quad
	          同\quad 計画 三郎\textsuperscript{*3}
             }
\foreignauthor{KOUZOU Ichirou,
			             KANKYO Jirou,
	                     and KEIKAKU  Saburo
             }%英文での著者表記
\affiliation{
	\mbox{*1}&建築工業大学工学部建築学科 教授・工博 &  Prof., Dept. of Architecture, Faculty of Engineering,\\
	&&Kenchiku Institute of Technology, Dr. Eng.\\
	\mbox{*2}&建築工業大学工学部建築学科 助手・工修 & Research Assoc. Dept. of Architecture Faculty of Engineering\\
	&&Kenchiku Institute of Technology M. Eng.\\
	\mbox{*3}&建築工業大学工学部建築学科 大学院生・工修 & Graduate Student Dept. of Architecture Faculty of Engineering\\
	&&Kenchiku Institute of Technology M. Eng.
}%アスタリスクは本文で打っても数式で打っても文字間などうまくいかないので,上付き文字にして出力している。
%\ekeyword{Diversity Favorability, Single-view-point townscape, 4-view-points townscape, Station plaza}
\category{2.構造―2.振動―c.地盤震動}
\jkeyword{多様性,好ましさ,単一視点景観,4視点景観,駅前広場,レパートリー}
%%%---definition begin---%%%
\newcommand{\marker}[1]{\underline{\textbf #1}}

%%%---definition end---%%%



%ここから本文
\begin{document}
\maketitle %abstractの後にかいて,abstractを段組にしない
\pagestyle{fancy}%draftの場合,ヘッダーを表示するとよい。
%\pagestyle{empty}%emptyにするとヘッダーは表示されない。
%\tableofcontents %目次


%\section{序論}
%	\subsection{背景}
%	\url{http://www.aij.or.jp/jpn/transaction/ronbun/}
%
234567890
1234567890
123\\2\\3\\4\\5\\6\\7\\8\\9\\10\\
1\\2\\3\\4\\5\\6\\7\\8\\9\\20\\
1\\2\\3\\4\\5\\6\\7\\8\\9\\30\\
1\\2\\33\\
あいうえおかきくけこさしすせそたちつてとなにぬ\\2\\3\\4\\5\\6\\7\\8\\9\\10\\
1\\2\\3\\4\\5\\6\\7\\8\\9\\20\\
1\\2\\3\\4\\5\\6\\7\\8\\9\\30\\
1\\2\\33\\
1\\2\\3\\4\\5\\6\\7\\8\\9\\10\\
1\\2\\3\\4\\5\\6\\7\\8\\9\\20\\
1\\2\\3\\4\\5\\6\\7\\8\\9\\30\\
1\\2\\3\\4\\5\\6\\7\\8\\9\\40\\
41\\
1\\2\\3\\4\\5\\6\\7\\8\\9\\10\\
1\\2\\3\\4\\5\\6\\7\\8\\9\\20\\
1\\2\\3\\4\\5\\6\\7\\8\\9\\30\\
1\\2\\3\\4\\5\\6\\7\\8\\9\\40\\
41\\

\section{原稿書式について}

	\subsection{入手方法}
	ホームページに掲載されている「原稿書式テンプレート」を各自ダウンロードし、必ず用いてください。
	研究報告集はオフセット印刷のため所定の書式以外の原稿は受け付けません。
	(ご注意!)必ず当該年度の原稿書式テンプレートを使用してください。
	\subsection{原稿頁数}
	1題は必ず4頁としてください。
	\subsection{字数}
	下記の字数枠に従って本文、図表などを納めてください	
	第1頁:1,518字(23字×33行×2欄) \\
	第2~3頁:1,886字(23字×41行×2欄) \\
	第4頁: 1,748 字(23 字×38 行×2 欄)\\
	原稿作成には、コンピュータの使用を原則とします。 コンピュータによってテンプレートに若干の差が生じる場合は、書式を上マージン 30mm 程度、 下マージン 20mm 程度、左右マージン 17mm 程度、本文は2段組とし1段を幅 84 mm程度(段 の間隔5mm程度)に設定し、字数を守り作成してください。
\section{文字フォントについて}
	\begin{enumerate}
		\item 本文の文字は黒色としますが、図・表・写真等については色の制限はありません。
		ただし、冊子版 研究報告集へは提出された PDF ファイルをモノクロプリントしたものを版
		下にして印刷しますので、色によっては色調が鮮明にでない場合があります。
		\item 提出された原稿(PDF ファイル)は、そのままの形式で CD-ROM 版 研究報告集へ収録します
		ので、以下のフォントを使用してください。
		\begin{table*}[bhp]
			\centering
			\begin{tabular}{l|p{16\zw}p{16\zw}}
				OS & Windows & Macintosh(OS 10.2 以上)\\ \hline
				日本語フォント & MS 明朝、MS ゴシック &  MS 明朝、MS ゴシック、ヒラギノ\\
				英字フォント & Arial, Century, Helvetica, Symbol, Times, Times New Roman & 同左
			\end{tabular}
			\caption{どうしても他のフォントを使用したい場合は、PDF 作成時にフォントの埋込みを行ってください。}
		\end{table*}
		\item コンピュータの機種により文字化けが発生する可能性がありますので、漢字コードは第二水 準以内の文字をお使いください。特に Windows をお使いの場合で、人名辞典にある「髙」や 「﨑」など、第二水準にはない文字がありますので注意してください。
	\end{enumerate}
	
	\section{以下は黄表紙のもの}
	\begin{enumerate}
		\renewcommand{\labelenumi}{$ \textcircled{\scriptsize\arabic{enumi}} $}%ここだけ箇条書きの数字を変更する
		\item 先のタイトル 14pt MS明朝 英数はCentury 英文の場合はすべて大文字/サブタイトル 10.5pt MS明朝 英数はCentury 中央揃え。英文の場合は最初の1語のみキャピタルラージとしています。
		
		\item 下行のタイトル 10.5pt MS明朝 英数はCentury 英文の場合はすべて大文字/サブタイトル 9pt MS明朝 英数はCentury 中央揃え。英文の場合は最初の1語のみキャピタルラージとしています。
		
		\item 和文著者名 10.5pt MS明朝 文字均等割付5字,氏名が3字以下は名字と名前の間に全角スペースを入れて均等割付5字,5字以上は名字と名前の間に半角スペースを入れてください。中央揃え,*(合い印)は全角*を上付きにしてください。
		
		\item 英文著者名(\textit{Namae MYOUZI})  10.5pt \textit{Century Italic}  
		
		\item 英文要旨(Abstract) 8pt Century 行間1行(14pt相当) ,左右インデント 各4字。	
		
		\item {\timesnewroman\textit {\textbf{Keywords:~}}} の見出しは9pt {\timesnewroman\textit {\textbf{Times New Roman Italic Bold}}}としてください。

		\item 英文キーワードは 8pt Times New Roman Italic 行間1行(14pt相当) 左右インデント 各4字 中央揃え。

		\item 和文キーワードは 8pt MS明朝 行間1行(14pt相当) 左右インデント 各2字 中央揃え。
		
	\end{enumerate}

\section{質疑討論(回答)の書き方について}
「版下原稿執筆の手引き」(組見本B)を参照してください。
質疑討論では,対象とする論文の著者名・論文名・号数・掲載年月日を先のタイトル,下行のタイトルともにカッコ付きサブタイトルとして記載してください。回答では、英文サブタイトルのみに記載してください。

\section{見出し}
(1) 見出しは 8pt MSゴシック 英文はArial

\section{第1頁の脚注について}
脚注は組体裁の都合上Wordの「脚注」機能を使わずに表を使用して,罫なしで作成しています。
(1) 和文
文字の大きさは 7pt  MS明朝 行間=固定値10pt
(2) 英文
文字の大きさは 7pt Century  行間=固定値10pt
(3) 表の「列」幅
本文が和文の場合は,表の「列」幅は和文(左側 70.5㎜),英文(右側 97.5㎜)としています。本文が英文の場合は,左右入れ替えます。英文(左側 97.5㎜),和文(右側 70.5㎜)としています。

\section{表について}
表と本文の間は1行空け中央揃えにしてください。
表題は,表の上に罫なしのセルを作りその中に入れると,表と表題がバラバラになりません。
表番および表題はMSゴシック(英文はArial)で英語表記を推奨する。

\begin{table}[H]
\caption{The target station square}
\small
\centering
\begin{tabular}{p{35mm}|p{35mm}}
	\hline 
	A name of the station and an exit name
 & The form of an open space
\\ 
	\hline \hline
	Chuo-rinkan & Rotary\\
	Saginuma & Rotary\\
	Tsukimino & Rotary\\
	Miyazaki-dai& Pedestrian space\\
	Minami-Machida& Pedestrian space\\
	Futako-tamagawaen (west)&Pedestrian space\\
	Suzukake-dai&Pedestrian space\\
	Jiyugaoka&Rotary\\
	Tsukushino&Pedestrian space\\
	Yutenji&Rotary\\
	Nagatsuta&Rotary\\
	Den-enchofu&Rotary\\
	Ichigao&Pedestrian space\\
	JR Musashikosugi&Rotary\\
	Eda&Rotary\\
	Tokyu Musashikosugi&	Pedestrian space\\
	Azamino (east)&Pedestrian space\\
	Hiyoshi (south)&Rotary\\
	Azamino (west)&Rotary\\
	Sakuragicho&Pedestrian space\\
	Tama-plaza&Pedestrian space\\
	Ikegami&Rotary\\
	\hline 
\end{tabular} 
\end{table}

\section{図について}
本文と図(\figref{fig:figure1})の間は1行空け中央揃えにしてください。
図番および図題はMSゴシック(英文はArial)で英語表記を推奨する。
\begin{figure}[h]
	\centering
	\includegraphics[width=1\linewidth]{../figure/figure1}
	\caption{Photography point}
	\label{fig:figure1}
\end{figure}


\section{写真について}
 写真(\figref{fig:photo1})と本文の間は1行空け中央揃えにしてください。
写番および写題はMSゴシック(英文はArial)で英語表記を推奨する。

\begin{figure}[h]
	\centering
	\includegraphics[width=1\linewidth]{../figure/photo1}
	\caption{An example of the photograph B}
	\label{fig:photo1}
\end{figure}


\section{カッコ付き}
箇条書き番号はカッコと数字を半角にして,カッコ・数字ともに書体をMS明朝にするときれいに揃います。
 例:(1) 1) 1)(右側の片カッコの1は全角)
\footnote{注表題は 7pt MSゴシック(英文はArial) 上1行アキ}
\footnote{注の文字の大きさは 7pt MS明朝(英文はCentury) 行間 10.5pt 複数行は1字下げてください。番号は片カッコ付きで数字は半角にしてください。}

\section{英文要約または和文要約について}
 英文要約(600語以内)は本文の書体で論文の末尾に改頁し付ける。和文要約(3,000字以内)は本文の書体で論文の末尾に付ける。

\vspace{1\Cvs}
\begin{thebibliography}{9}
	\bibitem{参考文献1}
		参考文献表題は 7pt MSゴシック(英文はArial) 上1行アキ
	\bibitem{参考文献2}
		参考文献の文字の大きさは 7pt MS明朝(英文はCentury) 行間10.5pt 複数行は1字下げてください。番号は片カッコ付きで数字は半角にしてください。

\end{thebibliography}
\vspace{1\Cvs}
%後注を出力
\theendnotes
\makeaffiliation
\end{document}          
