% written by Kei\right)  YASUDA 2018
% 応募要領指定フォントに準ずるために,xelatexを用いる。
\documentclass[base=10pt,magstyle=real,a4paper,twocolumn,xelatex,pandoc,jafont=ms]{bxjsarticle}
\usepackage{aij_taikai_xelatex}

%タイトル,著者,キーワードなどを記入
\title{タイトル}
\subtitle{サブタイトル}
\foreigntitle{Instrction for Summaries of Technical Papers of Annual Meeting}
\foreignsubtitle{Sub title}
\author{
	\begin{tabular}{ccl}
		正会員 & $\bigcirc$ & 建築 太郎\textsuperscript{*} \\
		同 & \hspace{1em} & 建築 花子\textsuperscript{**}
	\end{tabular}
}
\foreignauthor{KENCHIKU Taro, KENCHIKU Hanako
             }%英文での著者表記
\affiliation{
	\mbox{*}&建築大学工学部 教授・工博 & Prof., Faculty of Eng., Architectural Engineering Univ., Dr. Eng.\\
	\mbox{**}&株式会社建築学会研究所 工修 & Architectural Technology , Inc., M. Eng.\\
}%アスタリスクは本文で打っても数式で打っても文字間などうまくいかないので,上付き文字にして出力している。
%\ekeyword{Diversity Favorability, Single-view-point townscape, 4-view-points townscape, Station plaza}
%\category{2.構造―2.振動―c.地盤震動}
\jkeyword{
	\begin{tabular}{lll}
		キーワード1 & キーワード2 & キーワード3 \\ キーワード4 & キーワード5 & キーワード6
	\end{tabular}
}
%%%---definition begin---%%%
\newcommand{\marker}[1]{\underline{\textbf #1}}
\renewcommand{\theenumi}{\alph{enumi}}
\renewcommand{\labelenumi}{\theenumi.}
%%%---definition end---%%%



%ここから本文
\begin{document}
\maketitle %abstractの後にかいて,abstractを段組にしない
%\pagestyle{fancy}%draftの場合,ヘッダーを表示するとよい。
%\pagestyle{empty}%emptyにするとヘッダーは表示されない。
%\tableofcontents %目次


%\section{序論}
%	\subsection{背景}
%	\url{http://www.aij.or.jp/jpn/transaction/ronbun/}
%
%234567890
%1234567890
%123\\2\\3\\4\\5\\6\\7\\8\\9\\10\\
%1\\2\\3\\4\\5\\6\\7\\8\\9\\20\\
%1\\2\\3\\4\\5\\6\\7\\8\\9\\30\\
%1\\2\\33\\
%あいうえおかきくけこさしすせそたちつてとなにぬ\\2\\3\\4\\5\\6\\7\\8\\9\\10\\
%1\\2\\3\\4\\5\\6\\7\\8\\9\\20\\
%1\\2\\3\\4\\5\\6\\7\\8\\9\\30\\
%1\\2\\33\\
%1\\2\\3\\4\\5\\6\\7\\8\\9\\10\\
%1\\2\\3\\4\\5\\6\\7\\8\\9\\20\\
%1\\2\\3\\4\\5\\6\\7\\8\\9\\30\\
%1\\2\\3\\4\\5\\6\\7\\8\\9\\40\\
%41\\
%1\\2\\3\\4\\5\\6\\7\\8\\9\\10\\
%1\\2\\3\\4\\5\\6\\7\\8\\9\\20\\
%1\\2\\3\\4\\5\\6\\7\\8\\9\\30\\
%1\\2\\3\\4\\5\\6\\7\\8\\9\\40\\
%41\\

\textbf{応募規定}
\section{研究内容}
	建築に関する学術・技術・芸術の最近の研究成果、または統計的資料、調査報告等で未発表のもの。
	ただし、「論文集」および本会「支部研究発表会」で発表したもの、ならびに他学会論文集等に発表したもので、特に建築に関連の深いものはこの限りでない。

\section{応募・発表資格}
	研究発表者の資格は次による。
	\subsection{講演発表者( $\bigcirc$ 印)は次のいずれかの条件を備えた者とする。}
		\begin{enumerate}
			\item 論文予約会員となっている正会員(個人)・準会員
			\item 論文予約会員以外の正会員(個人)・準会員であって、発表登録費を期限までに納入した者
		\end{enumerate}
	\subsection{共同発表者(連名者)は次のいずれかの条件を備えた者とする。}
		\begin{enumerate}
			\item 論文予約会員となっている正会員(個人)・準会員
			\item 論文予約会員以外の正会員(個人)・準会員であって、発表登録費を期限までに納入した者
			\item 会員外であって、発表登録費を期限までに納入した者
		\end{enumerate}
	
\section{発表者の権利・義務}
	研究発表者の権利、義務は次のとおりとする。
 		\begin{enumerate}
			\item 講演発表者($\bigcirc$印)は、大会学術講演会に出席し、1名1題に限り講演発表を行うことができる。
			\item 講演発表者は必ず大会に参加し講演発表をしなければならない。共同発表者の代理講演は認めない。
			\item 共同発表者(連名者)として名を連ねる論文の題数に制限はないが、共同発表者は講演発表を行うことはできない。
		\end{enumerate}
			

\section{応募方法}
	本会所定の研究発表申込書・研究発表梗概原稿の本会学術推進委員会への提出をもって申込みとする。
	なお、締切後の原稿の訂正は一切認めない。
	
\section{発表部門}
	発表部門は下記の12部門のいずれかとし、研究発表申込書の所定欄に発表希望部門・細分類・細々分類を明記する。
	1.材料施工、2.構造、3.防火、4.環境工学、
	5.建築計画、6.農村計画、7.都市計画、8.建築社会システム、
	9.建築歴史・意匠(建築論を含む)、10.海洋建築、11.情報システム技術、13.教育。
	\begin{enumerate}
		\item 発表部門:研究発表申込書の所定欄に発表希望部門・細分類および細々分類を明示する。ただし、プログラム編成に際し、学術推進委員会において発表部門を変更する場合がある。
		\item ポスター発表の決定は学術推進委員会プログラム編成会議が行う。発表方法等の詳細については、各発表部門を担当する調査研究専門委員会から連絡する。なお、ポスター発表は希望者以外にも依頼する場合がある。
	\end{enumerate}

\section{発表方法}
発表の方法は口頭発表(オーガナイズドセッションを含む)とポスター発表(ポスターセッション)の二種類とする。
ポスター発表の決定は学術推進委員会プログラム編成会議が行う。
発表方法等の詳細については、各発表部門を担当する調査研究専門委員会から連絡する。
口頭発表に際しては会場に備え付けられた機器(OHP)以外の使用はできない。

\section{研究発表梗概の採否}
研究発表梗概の採否は、学術推進委員会が決定する。
下記条項等に照らし大会学術講演会発表梗概として不適当と認められるものは採択しない。
不採択となった場合は本人に通知する。
なお、提出された原稿は一切返却しない。
\begin{enumerate}
\item 梗概に記した説明が著しく不十分なもの。
\item 同一または類似の研究発表がすでに行われているもの。ただし、既発表の研究内容を前進させたものはこの限りでない。
\item 内容が商業宣伝に偏したもの(商品名の使用には注意すること)。
\item 他者を誹謗中傷する内容を含むもの
\item 応募規程、執筆要領に反するもの
\end{enumerate}

\section{発表登録費}
発表登録費は表—1による。
ただし、留学生の会員の発表登録費は応募申込み時に申請があれば減額する。
\begin{enumerate}
	\item 論文予約会員は発表登録費を免除する。
	\item 数題にわたって応募する場合も発表登録費は1題分とする。
	\item 講演発表者($\bigcirc$印)は発表者を代表して共同発表者の納入の義務を果たすものとし、全員の発表登録費の納入のない場合は、当該研究の講演発表を停止する。
\end{enumerate}

\section{大会参加費}
	参加者は大会参加費(発表登録費とは別)を納入するものとする。
\section{梗概集}
	採択された研究発表梗概はDVD-ROM版梗概集および冊子版梗概集に収録し、「大会学術講演梗概集」として頒布する。
	なお、発表登録費納入者には当該発表梗概が掲載されているDVD-ROM版梗概集および冊子版梗概集1冊を無償で送付する。
	
\section{著作権}
	\begin{enumerate}
		\renewcommand{\theenumi}{\arabic{enumi}}
		\renewcommand{\labelenumi}{(\theenumi)}
		\item 著者は、掲載された研究発表梗概の著作権を本会に委託する。ただし、本会は、第三者から文献等の複製・引用・転載に関する許諾の要請がある場合は、原著者に連絡し許諾の確認を行う。
		\item 著者が、自分の論文を自らの用途のために使用することについての制限はない。なお、掲載された論文をそのまま他の著作物に転載する場合は、出版権に関わるので本会に申し出る。
		\item 編集出版権は、本会に帰属する。
	\end{enumerate}


\textgt{応募上の注意}
\section{入会手続き}
新規に本会会員・論文予約会員になろうとする者は、入会手続を済ませてください。
会員資格がないと応募できませんので十分ご注意ください。\\
会費等は郵便振替で送金してください。\\
口座番号:~00180-8-17187\\
口座名義:~社団法人日本建築学会

\section{発表登録費}
	\begin{enumerate}
		\item 論文予約会員以外の研究発表者(講演発表者ならびに共同発表者)は、発表登録費を納入しなければなりません。
		\item 発表登録費は採択決定後、論文予約会員以外の方に請求書を送付しますので、請求書に記載の期限までに納入してください。
		応募の際に小切手・為替等は一切同封しないでください。
		\item 留学生の会員は、研究発表申込書の所定欄に会員番号とともに「留学生」と明記してください。
	\end{enumerate}

\section{電子投稿による研究発表梗概原稿}
	\begin{enumerate}
		\item 電子投稿による研究発表梗概原稿はAdobe Acrobat 7.0以降(または同等品)を用いて変換したPDFファイルのみとします。
		Adobe Acrobatをお持ちでない方はあらたに購入していただく必要があります。
		PDFファイル読みとり専用のAcrobat ReaderではPDFファイルに変換することはできません。
		なお、原稿作成に使用するワードプロセッサー等のアプリケーション、OSは問いません。
		\item 原稿の提出はインターネット経由に限ります。
		フロッピーディスク等の郵送、持参は受け付けません。
		\item 原稿のPDFファイルの大きさは600KB以下としますが、できるかぎり300KB以下となるようご協力をお願いします。
		600KB以上のファイルはサーバーが受け付けませんのでご注意ください。
		なお圧縮ツールは使用しないでください。
		\item 本文の文字は黒色としますが、図・表・写真等については色の制限はありません。
		ただし、冊子版梗概集は、提出されたPDFファイルをモノクロプリンターでプリントしたものを版下にして印刷します。
	\end{enumerate}

\section{紙面投稿による研究発表梗概原稿}
	\begin{enumerate}
		\item 原稿は1件ごとに研究発表申込書が表紙となるように左上肩をホチキス止めし、所定の応募期間内に本会学術推進委員会あてに郵送、または本部事務局に持参してください。
		郵送する場合は「消印」が確認できる方法とし、「後納」「別納」等による方法は不可とします。
		応募期限に遅れた原稿は一切受理しません。
		なお、各支部では原稿などの提出を受け付けておりません。
		\item 原稿はそのままスキャナーでモノクロの図版としてPDFファイル化し、CD-ROM版梗概集に収録します。
		冊子版梗概集はPDFファイル化された原稿をモノクロプリンターでプリントしたものを版下にして印刷します。
	\end{enumerate}
\begin{table}
	\centering
	\caption{発表登録費}
	\begin{tabular}{lrcc}
		種別&発表登録費&講演発表&共同発表\\ \hline
		論文予約会員&免除&可&可\\ \hline
		正会員・準会員&9,000円&可&可\\ \hline
		留学生の会員&2,000円&可&可\\ \hline
		会員外&13,000円&不可&可
	\end{tabular}
\end{table}
\section{電子的な発表申込みと郵送による紙面投稿}
原稿は電子的な発表申込みの「講演申込登録完了」のページをプリントアウトしたものが表紙となるように1件ごとに左上肩をホチキス止めし、所定の応募期間内に本会学術推進委員会あてに郵送、または本部事務局に持参してください。
提出の条件は紙面投稿に準じます。

\section{オーガナイズドセッション}
オーガナイズドセッションに採用された研究発表者は、この応募規程により研究発表梗概原稿を提出してください。
オーガナイズドセッション発表原稿募集要領はこちらをご覧ください。

\textgt{ポスターセッション発表要領}
\section{ポスターセッションの申込み}
研究発表梗概原稿と研究発表申込書の提出をもって申込みとする。紙面投稿の場合は研究発表申込書の下段、目次原稿のポスターセッション(PS)欄に「ポ」と記入し、電子投稿の場合は発表申込書のポスターセッション欄をチェックする。


%\vspace{1\Cvs}
%\begin{thebibliography}{9}
%%	\bibitem{参考文献1}
%%		参考文献表題は 7pt MSゴシック(英文はArial) 上1行アキ
%\end{thebibliography}
%\vspace{1\Cvs}
%後注を出力
%\theendnotes
\makeaffiliation
\end{document}          
